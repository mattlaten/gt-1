\documentclass[a4paper]{article}

\usepackage{marvosym}
\usepackage{url,parskip}			%formatting

%better formatting of the A4 page
\usepackage{fullpage}
%An alternative to Layaureo can be usepackage{fullpage}
 
\usepackage{supertabular} 		%for Grades
\usepackage{titlesec}			%custom section

\usepackage{enumitem}
 
%Setup hyperref package, and colours for links
\usepackage{hyperref}

\begin{document}

\title{Graph Theory Assignment 1}
\author {
    Davies, Michael\\
    \and
    Goosen, Chris\\
    \and
    Laten, Matthew\\
    \and
    Sharwood, Bee\\
    \and
}
\maketitle

\textbf{1.2 d)}
Firstly, we notice that to get from $Q_{t-1}$ to $Q_t$, we take the cartesian
product of $Q_{t-1}$ with $K_2$. This results in the number of vertices in
$Q_{t-1}$ being doubled. Since $Q_1 = K_1$ and the order of $Q_1$ is 2, we have
that the order of $Q_t$ is given by $2 \times 2 \times ... \times 2$ $t$ times,
or rather $ord(Q_t) = 2^t$.

To compute the size of $Q_t$, we interpret each vertex as a t-digit binary
string, which is adjacent to every binary string which differs from it by
exactly one place. Further, we notice that each vertex has degree $t$. Thus,
the sum of the degrees of $Q_t$ is given by
\begin{equation}
    \sum_{v \in V(Q_t)} deg(v) = t\cdot2^t
\end{equation}
Thus, the size is $\frac{t\cdot2^t}{2} = t\cdot2^t$.

\textbf{1.4)}
We are looking for a graph which is cubic or 3-regular with order 7. However,
we know that the sum of the degrees for a graph has to be even. If such a graph
were to exist, the sum of the degrees would be $3 \times 7 = 21$ which is odd. 
Hence, no such graph exists.

\textbf{1.5)}
For a graph of degree $n$, the possible degrees a vertex in the graph can have
are $n-1, n-2, ..., 1, 0$. For an irregular such graph, each vertex must have
a distinct degree, and so each degree must be used exactly once. Thus, we get that
the vertex with degree $n-1$ must be adjacent to all other vertices, including
the vertex of degree $0$. The only situation where this is possible, is if
$n-1 = 0 \Rightarrow n = 1$ so there is only 1 vertex i.e. the graph is trivial.

\end{document}
